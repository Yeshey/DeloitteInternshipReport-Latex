    \subsection{Análise de Incidentes}\label{sub:incidentes}

        Incidentes, ao contrário de \textit{defects}, têm uma certa urgência associada, devido a serem um problema reportado diretamente por um utilizador, havendo, normalmente, alguém à espera que este seja resolvido. Pelo que se recorre muito mais a \textit{datafixes} para desbloquear utilizadores e nem sempre é possível ou viável chegar à causa, ou raiz dos problemas. A Tabela \ref{tab:incidentes_trabalhados} apresenta uma lista com todos os incidentes trabalhados.

        \begin{table}[htbp] % htbp
            \centering
            \begin{tblr}{
                % example for tblr: https://tex.stackexchange.com/questions/603349/tabularray-and-new-command-for-multicolumn-cells
                % another example: https://tex.stackexchange.com/questions/605676/tabularray-how-to-control-the-vertical-alignment-of-the-cells-contents
                hlines={lightgray}, vlines={lightgray},
                width = \linewidth,% total width set to width available
                %rows = {c,m}, % c aligns horizontally, m aligns vertically, aligns all rows
                colspec={X[1.4,c,m] X[1.4,c,m] X[3,c,m] X[0.9,c,m]},
            }
            % \textbf{Tempo trabalhado}
            \textbf{Título do incidente} & \textbf{Incidente em Português} & \textbf{Descrição} & \textbf{Prioridade} \\

            Endorsement showing ``not Responded'' & \textit{Endorsement} mostrando ``not Responded'' & \textit{Endorsement} enviado mostrava ``Not Responded'' apesar de ter sido respondido, a solução foi apagar e pedir ao utilizador para o recriar & Média \\
            ``Unable to submit cancelation'' & Não foi possível enviar o cancelamento & Refira à Secção \ref{secsec:inc0065686} & Média \\
            ``Cannot Assign Roles'' & Não é possível atribuir os \textit{roles} & O botão de ``Assign Roles'' não se encontrava visível. Através da chamada com uma colega e de uma chamada com o utilizador, percebemos que o problema era o botão ``Add subpanels'' na aba dos \textit{Underwriters} que estava ativo, e nesse estado, escondia o botão ``Assign Roles'' & Média \\
            ``error NM0011234 to lineslip FNM0003923 --- All selections invalid'' & erro NM0011234 para \textit{lineslip} FNM0003923 --- Todas as seleções são inválidas & Refira à Secção \ref{secsec:inc0064225} & Média \\

            \end{tblr}
            \caption{Incidentes Trabalhados}
            \label{tab:incidentes_trabalhados}
            \source{Documentação Interna} % 1- https://impalaintech.com/blog/mendix-vs-outsystems-vs-appian/
        \end{table}

        % ``O meu primeiro Incident'' No onenote INC0063053

        \subsubsection{INC0064225 - Todas as seleções são inválidas}\label{secsec:inc0064225} % INC0064225 com o marcio

            Incidente original: ``\textit{All selections invalid}''

            % OMG INCIDENTE INC0064225, aquele que fizeste datafix com o marcio

            \begin{table}[H] % htbp
                \centering
                \begin{tabularx}{1\textwidth}{|>{\raggedright\arraybackslash}X|}
                    \hline
                    \rowcolor{lightgray}
                    \textbf{Incidente INC0064225} \\
                    \hline
                    \rowcolor{lightgray!20}
                
                    \textbf{Descrição do Incidente:} O utilizador ficou bloqueado ao submeter uma nova Firm Order. Depois de fazer o processo todo de submissão como esperado, aparecia o erro ``All selections are invalid, Please review the breakdown below and select Cancel to return to the Submission Pack'', apresentando por baixo uma lista com todos os UWs a dizer estarem inativos.

                    \\
                    \hline
                \end{tabularx}
                \caption{Incidente INC0064225}\label{table:usincINC0064225}
                \source{Resumo da Informação do Incidente no Service Now}
            \end{table}

            A descrição do incidente encontra-se na Tabela \ref{table:usincINC0064225}.

            Durante a investigação, verificou-se um possível equívoco no número fornecido pelo utilizador ``NM0011234'', que se presumiu que deveria ser ``NM0011224''. 

            Identificou-se inicialmente a possibilidade de resolver o problema removendo e adicionando novamente a \textit{Master Facility} à declaração. Esta abordagem foi sugerida ao utilizador, como uma solução potencial para o problema.
            
            Ao analisar as participações associadas à \textit{Facility}, notou-se ainda uma discrepância numa delas, relacionada com um utilizador chamado ``John''. Embora não pareça ser a causa direta do problema, esta discrepância foi registada para investigação adicional, caso a solução inicial não fosse bem-sucedida.

            Foi então mandado o pedido ao utilizador para adicionar a MF de novo. Recebeu-se resposta negativa atempadamente, informando-nos que isto já fora tentado e não funcionara.
            
            No entanto, na base de dados era possível ver que a última mudança ao contrato fora há mais de uma semana atrás, foi decidido, portanto, fazer a \textit{datafix} para corrigir o utilizador ``John'' e marcar uma reunião com o utilizador. Devido ao tempo que o \textit{datafix} levou a ser aplicado, acabou por não se conseguir falar com o utilizador antes do final da semana. Mas mais tarde recebeu-se a notícia que o \textit{datafix} efetuado fora suficiente para desbloquear o utilizador e o incidente pôde ser fechado.
            
        \subsubsection{INC0065686 - Não foi possível enviar o cancelamento}\label{secsec:inc0065686} % INC0064225 com a ines
                
            Incidente original: ``\textit{Unable to submit cancelation}''

            \begin{table}[H] % htbp
                \centering
                \begin{tabularx}{1\textwidth}{|>{\raggedright\arraybackslash}X|}
                    \hline
                    \rowcolor{lightgray}
                    \textbf{Incidente INC0065686} \\
                    \hline
                    \rowcolor{lightgray!20}
                
                    \textbf{Descrição do Incidente:} O utilizador mandou uma Firm Order para um UW que não devia ter mandado, por isso clicou em ``withdraw'' do pedido, mas o UW conseguiu aceitar. Ao ver que o UW tinha aceite, foi mandado um pedido de cancelamento, mas o UW não conseguia visualizá-lo, e ao abrir a submissão deste pedido, o programa fica parado a carregar infinitamente.

                    \\
                    \hline
                \end{tabularx}
                \caption{Incidente INC0065686}\label{table:incINC0065686}
                \source{Resumo da Informação do Incidente no Service Now}
            \end{table}

            A descrição do incidente encontra-se na Tabela \ref{table:incINC0065686}.

            No decorrer da análise do incidente INC0065686, foi identificado o problema na base de dados. Existe uma sequência de documentos criados na DB de MongoDB que são criados quando o estado de um UW é mudado, estas são as negotiations. Depois de se comunicar com um membro da equipa familiarizado com o funcionamento desta coleção, apercebeu-se que depois da negotiation com o estado ``withdrawn'' foi gerada subsequentemente uma negotiation com o estado \texttt{pending\_unconditional\_line}, que significa que o pedido tinha sido enviado para o UW mesmo tendo sido withdrawn; se fosse um novo pedido e não o mesmo, teria que haver uma negotiation \texttt{request\_for\_line\_or\_binder} antes da \texttt{pending\_unconditional\_line}.

            Não foi encontrada nenhuma indicação na Base de Dados do pedido de cancelamento descrito pelo utilizador.

            Depois de alguma discussão decidiu-se que a solução seria fazer um ``soft delete'' das negociations depois do withdraw. Para manter a integridade da base de dados foi necessário também alterar a coleção das ``participations'', que continha informação àcerca do UW e da sua participação atual no contrato, tendo sido necessário remover cinco campos que tinham sido inadvertidamente adicionados devido às negotiations erróneas que foram criadas, e alterado o campo status da participation para ``new''.
            
            No fim foi criado um \textit{defect} com os passos para reproduzir que estavam por definir, onde se detalhou o problema: ``UW capaz de aceitar pedido withdrawn''.

            % INC INC0065686 (aquele que viste com a inês e foi preciso mandar datafix para remover as negotiations)

            % Incidente do Craveiro no onenote
            %  aquele que falaste com o user com a andereia, e que o problema era os subpanels