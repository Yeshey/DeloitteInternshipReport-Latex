\subsubsection{Service Studio}

    O Service Studio é o ambiente de desenvolvimento low-code de OutSystems, é onde se criam as aplicações, módulos e a lógica por detrás destas, bem como o desenvolvimento das interfaces de utilizadores para aplicações, quer sejam web tradicionais, web reativas ou móveis. 
    É neste ambiente também que se definem os modelos das bases de dados internas a OS. 
    
    É uma ferramenta muito importante no contexto dário para o desenvolvimento e manutenção cotínua da aplicação.

\subsubsection{Service Center}

    A plataforma oferece uma variedade de configurações que podem ser ajustadas para personalizar o comportamento em diferentes áreas. Estas configurações são ajustáveis a partir do Service Center, permitindo a visualização e gestão de uma panóplia de funcionalidades como módulos e dependências entre eles, gestão das diferentes aplicações do projeto, gestão do versionamento de cada módulo através da publicação da versão desejada. Bem como o ajuste de variáveis globais.
    
    O Service Center é também uma ferramenta essencial para gerir e diagnosticar erros durante o desenvolvimento ou manutenção, permitindo visualizar logs relacionados com timers, processos, módulos, \textit{screen requests}, \textit{service actions}, \textit{traditional web requests}, integrações, extensões ou e-mails.

\subsubsection{Lifetime}

    O Lifetime é uma plataforma do ecossistema de OutSystems que gere o ciclo de vida completo das aplicações. Proporciona visibilidade total sobre os diferentes ambientes, desde o desenvolvimento até à produção, sendo a ferramenta principal na realização de \textit{Deployments}, resolvendo conflitos e efetuando \textit{merges} automaticamente sempre que possível e ajudando o utilizador no processo em caso de conflitos. Oferece, além disso, funcionalidades para versionamento e migração.