\section{Conclusão}\label{sec:conclusao}

        Neste capítulo, vai ser realizada uma reflexão sobre o trabalho desenvolvido durante o estágio. Vão ser destacados desafios encontrados e avaliados os ganhos do estágio em relação ao que fora planeado inicialmente. Será também explorado o trabalho futuro do projeto e que direções este pode tomar.

    \subsection{Desafios encontrados e melhorias}\label{sub:desafios_encontrados}

        Ao longo do estágio, foram surgindo diversos desafios que exigiram uma abordagem proativa para serem superados de forma eficiente e atempada. Estas dificuldades, apesar de representarem obstáculos, proporcionaram valiosas oportunidades de aprendizagem e crescimento profissional.

        Entre os desafios enfrentados, podem destacar-se os 
        seguintes:

        \subsubsection*{Desafios de Adaptação:}
        A adaptação ao projeto é uma constante, devido à equipa de grande escala e ao constante e célere desenvolvimento da plataforma, é impossível estar confortável com todas as roldanas em funcionamento no projeto. Realçam-se, no entanto, os seguintes como sendo os obstáculos mais notórios, especialmente no início do estágio:
        \begin{enumerate}
            \item Adaptação a novas tecnologias;
            \item Adaptação aos processos de um grande projeto;
            \item Adaptação ao ambiente de uma grande empresa;
            % \item Adaptação aos prazos abstratos de tarefas com diferentes prioridades.
        \end{enumerate}
        
        \subsubsection*{Desafios de Comunicação:}
        A comunicação eficaz e partilha de ideias com uma, e mesmo, várias equipas que trabalham em conjunto é uma destreza difícil de se ensinar, pelo que o estágio oferece uma grande vantagem na possibilidade de desenvolver estas capacidades. Estas foram, e permanecerão sempre, um ponto de melhoria, que fora também mencionado nos check-ins mensais:
        \begin{enumerate}
            \item Comunicação eficaz com a equipa;
            \item Confiança na partilha de ideias;
            \item Iniciativa em intervenções adequadas e partilha de ideias.
        \end{enumerate}

        \subsubsection*{Outros Desafios}
        Para além dos já delineados, houve também outros obstáculos, nomeadamente:
        \begin{enumerate}
            \item Resolução de problemas técnicos inesperados;
            \item Gestão eficiente do tempo em tarefas multifacetadas;
            \item Distância entre o lugar de trabalho e instituição de ensino.
        \end{enumerate}

        A colaboração com membros da equipa, a comunicação eficiente, a orientação e o apoio recebidos tiveram um papel crucial na superação destes desafios. Neste contexto, a experiência proporcionada pelos desafios encontrados durante o estágio contribuiu significativamente para o desenvolvimento profissional, sendo estas competências de comunicação os autores principais na peça de preparação para enfrentar futuros desafios com confiança e determinação. 

    \subsection{Reflexões Finais}\label{sub:conclusoes}

        Durante o período de estágio, a experiência revelou-se mais educativa e árdua do que inicialmente prevista. A dinâmica entre a plataforma OutSystems e as exigências específicas da empresa de seguros proporcionou uma experiência enriquecedora, repleta de desafios e oportunidades de aprendizagem significativas.

        Entre as inúmeras oportunidades que surgiram ao longo do estágio, destacam-se:
        \begin{enumerate}
            \item Aprofundamento do conhecimento da plataforma OutSystems e outras tecnologias. A interação com as diversas tecnologias permitiram uma imersão profunda no desenvolvimento de aplicações ágeis, alinhando-se às necessidades específicas da empresa de seguros e proporcionando uma base sólida de conhecimento nas diversas ferramentas;
            \item Colaboração com uma equipa especializada. A oportunidade de trabalhar com uma equipa dedicada e conhecedora do setor de seguros e do setor de \textit{low-code}, possibilitou uma compreensão mais abrangente das complexidades e nuances destas áreas e dos processos de um trabalho empresarial;
            \item Resolução de desafios específicos da área. Os desafios enfrentados estiveram diretamente relacionados com as necessidades do setor de seguros, desde a gestão eficiente de dados sensíveis até à implementação e manutenção de processos mais ágeis e eficazes.
        \end{enumerate}

    \subsection{Trabalho Futuro}\label{sub:trabalho_futuro}

        O projeto atual tem já mais de dois anos com a Deloitte e é a segunda iteração da plataforma, vai sempre, no entanto, necessitar um esforço de manutenção, quer seja para atender a questões e pedidos dos utilizadores, quer para manter uma norma de segurança alta.

        O trabalho futuro vai basear-se essencialmente nas \textit{user stories} pedidas pelo cliente e problemas encontrados na plataforma, no entanto, em termos de arquitetura e processos, existem também vários pontos que podem ser melhorados.

        \textbf{Tarefas planeadas a ser executadas:}
        \begin{itemize}
            \item Criação de um API geral da plataforma de seguros para se poder interagir com esta programaticamente;
            \item Atualização e manutenção constante de dependências e ferramentas;
            \item Atendimento às necessidades e pedidos dos utilizadores;
            \item Melhoria e atualização da documentação.
        \end{itemize}

        \textbf{Tarefas que poderão ser realizadas:}
        \begin{itemize}
            \item Automatização parcial ou total da análise de processos de OutSystems que acabam suspensos ou em erro;
            \item Integração com o API em desenvolvimento da plataforma para automatização dos testes. 
        \end{itemize}

        