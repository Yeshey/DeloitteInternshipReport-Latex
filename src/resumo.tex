\section{Resumo}\label{sec:resumo}

    Este relatório descreve um trabalho onde se aborda a modernização das operações de seguros através da implementação de uma abordagem \textit{low-code}, com foco na plataforma OutSystems. Ao longo do estágio curricular, foram enfrentados desafios significativos, desde a adaptação a novas tecnologias até à integração eficiente de sistemas existentes. A análise abrangente do desenvolvimento \textit{low-code} multifacetado destaca os benefícios e as considerações essenciais ao utilizar esta abordagem no contexto específico das operações de seguros. 
    
    Este relatório visa contribuir para a compreensão do impacto do desenvolvimento \textit{low-code} na área de seguros, destacando as vantagens e desvantagens das várias tecnologias, os seus competidores e como se interligam entre si, nomeadamente Azure, MongoDB, Nuxeo e competidores.

    Toda a descrição do processo acontecerá num contexto de um estágio curricular cujo propósito foi o avanço académico do estagiário na área de Tecnologias de Informação e cujas tarefas estenderam-se entre a correção e reporte de defeitos da aplicação, a análise e acompanhamento dos utilizadores nos seus incidentes e a execução de terceiras tarefas como outras que tenham sido necessárias ou requisitadas, bem como a tentativa de constante automação e aprimoramento dos processos usados nas resoluções.  


    \medskip
    \noindent
    {\small{\bf Palavras-chave:} 
    outsystems, \textit{low-code}, seguros, seguradoras, mongodb, azure, nuxio, subscritor, corretor, contrato, atlas, service center, lifetime, service studio, ServiceNow}