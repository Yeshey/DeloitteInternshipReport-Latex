\section{Organização do Projeto e Empresa}\label{sec:org-projeto-empresa}

    Nesta secção será explorada a estrutura e organização da Deloitte e do projeto no que toca a recursos humanos, licenças, programas, e a integração com os restantes mecanismos de gestão da Deloitte PT. Devido à grande escala do projeto RIL este envolve uma colaboração extensa entre a Deloitte PT, a Deloitte UK e a Deloitte India. 

    \subsection{Estrutura Organizacional da Deloitte PT}
        
        A Deloitte Portugal organiza-se de forma a proporcionar uma integração eficiente dos novos contratados, facilitando o seu desenvolvimento de carreira e apoio contínuo aos colaboradores. Nesta secção, abordaremos as principais práticas e elementos relacionados à estrutura organizacional da Deloitte PT.
        
        \subsubsection{Formações e Integração}\label{formacoes_e_integracao}
        
            As formações, como já falado no primeiro capítulo, desempenham um papel crucial na integração de novos colaboradores à equipa em que são inseridos enquanto elevam simultaneamente  as capacidades dos novos recrutados a termos de ferramentas que são frequentemente usadas na empresa ou em projetos da empresa. Servem também como uma avaliação inicial das aptidões de cada ``new hire'' para mais facilmente identificar um projeto que se adeque a ele.
        
        \subsubsection{Hierarquia e Progressão de Carreira}

            Cada profissional tem um estatuto para com a hierarquia da empresa. Esta hierarquia dita quem tem poder de decisão sobre quem, quem é que faz os ``check-ins'', e reflete-se também no salário de cada um.

            \textbf{Check-Ins}: Os ``check-ins'' são reuniões mensais realizadas entre cada membro da Deloitte e o seu superior. Durante estas reuniões, são avaliados pontos fortes e fracos desde o último encontro, identificam-se áreas de melhoria e é dada uma nota. Estes encontros são essenciais para manter o desenvolvimento profissional positivo e saber quais são os pontos de melhoria dos profissionais, permitindo um contínuo aprimoramento das suas habilidades. São também estabelecidas metas, objetivos anuais ou mensais.
                        
            Existem vários estatutos hierárquicos, por ordem crescente de hierarquia, os seguintes:

            \begin{itemize}
                \item \textbf{Trainee} \\
                    Um estatuto inicial na Deloitte, reservado para aqueles que estão nos primeiros etapas de suas carreiras. Os trainees geralmente estão a ter a sua primeira experiência de trabalho na área ou estão a trabalhar no contexto de um estágio curricular ou profissional e estão em processo de aprendizagem e desenvolvimento;
                \item \textbf{Programmer} \\
                    A designação para quem foi contratado como programador e tenha pouca experiência, e um a dois anos;
                \item \textbf{Experienced Programmer} \\
                    Após algum tempo e feedback positivo, os Programmers são promovidos ao estatuto de Experienced Programmers;
                \item \textbf{Team Leader} \\
                    O estatuto de Team Leader é alcançado por indivíduos que estão na empresa há algum tempo e demonstraram já capacidade de liderança e tenham uma boa ideia dos funcionamentos internos da empresa. Nesta posição, assumem a responsabilidade por uma equipa, estando encarregados de realizar os Check-ins com os restantes membros.
                    
                    Podem também, a partir deste título, assumir o papel de ``Career Coach'' para outros membros da empresa;
                \item \textbf{Pro Manager} \\
                    Enquanto os papéis anteriores necessitavam ainda de um grande contacto com o projeto e as tarefas em si, os seguintes são papéis que envolvem maioritariamente gestão de recursos.
                    
                    Têm um papel-chave na gestão de equipas e projetos;
                \item \textbf{Manager} \\
                    Uma posição de gestão mais avançada, envolvendo mais responsabilidades significativas na coordenação e supervisão de projetos;
                \item \textbf{Senior Manager} \\
                    O último nível a que se pode ser promovido dentro da categoria de gestão com mais poder e responsabilidade na liderança estratégica dos projetos e gestão do cliente;

                \item \textbf{Associate Partner} \\
                    Uma posição associada já à gestão de colaborações da Deloitte, começam a ter responsabilidades em relação aos novos contratados e a que projetos é que são alocados, bem como o contacto e fecho de contratos com empresas parceiras;
                \item \textbf{Partner} \\
                    O estatuto mais elevado, com acrescidas responsabilidades de gestão de projetos e de empresas parceiras.
            \end{itemize}

        % Individuos que integram do programa BrightStart que permite realização do trabalho e de estudos simultaneamente.
        
        \subsubsection{Buddys}

            Buddys são um sistema da Deloitte para mais facilmente integrar novos contratados. Normalmente entram em contacto no início da contratação e servem como alguém com quem o individuo se possa sentir confortável a fazer perguntas gerais acerca da empresa, do funcionamento da empresa, da equipa, etc. Estão num nível hierárquico idêntico a quem são atribuídos, mas, estão na empresa à mais tempo. 

            São uma forma de fazer um recém-contratado se sentir mais bem integrado num ambiente novo.
            
        \subsubsection{Career Coaches}
        
            A cada novo elemento na empresa é também atribuído um ``Career Coach'', alguém com quem se possa falar sobre o ambiente na sua equipa ou como se sente em relação ao trabalho sem ser diretamente da mesma equipa. Serve também para aconselhar acerca da progressão de carreira na Deloitte, é normalmente marcada uma reunião mensal com o ``Carrear Coach'' atribuído.

        \subsubsection{Trabalhadores externos}

            O funcionamento dos trabalhadores externos na Deloitte segue um modelo de subcontratação, conhecido como \textit{outsourcing}. Enquanto os funcionários internos são contratados diretamente pela Deloitte, os trabalhadores externos são alocados aos projetos através de uma empresa contratada.
            
            A Deloitte estabelece contratos com empresas externas para fornecer profissionais que serão alocados a projetos específicos. Neste modelo, a Deloitte realiza pagamentos à empresa terceira pelos serviços prestados, e esta última é responsável por fornecer os recursos humanos necessários.
            
            Esta abordagem permite à Deloitte flexibilidade na gestão da sua frota de trabalho, garantindo a disponibilidade de especialistas conforme necessário para atender aos requisitos de projetos específicos. Ao mesmo tempo, as empresas externas fornecem profissionais especializados para atender às necessidades da Deloitte, criando uma colaboração eficaz no âmbito de \textit{outsourcing}. Este é de forma geral também o modelo de trabalho da Deloitte para com outras empresas.

        \subsubsection{Engagement e Timesheet}\label{subsub:enga_timesheet}

            A gestão de projetos na Deloitte PT é acompanhada de perto através de ferramentas como \textit{Engagement} e \textit{Timesheet}. A cada quinzena é obrigatório a todos os trabalhadores da Deloitte o preenchimento e a submissão da \textit{timesheet} com informações das horas trabalhadas bem como os \textit{engagements}, para ficar registado que tarefas ou projetos é que foram trabalhados.
            
            O sistema de \textit{Engagement} é fundamental para planear e oferecer uma visão geral da progressão de cada indivíduo, é também uma forma de comprovar aos parceiros que trabalho é feito bem como quantos trabalhadores e horas são debitadas em que projetos. É usado também para efeitos de faturação.

            Estas ferramentas não apenas auxiliam na alocação eficiente de recursos, mas também fornecem conhecimentos valiosos sobre a distribuição do esforço das equipas ao longo dos projetos. Esta transparência contribui para uma melhor compreensão no investimento de tempo em diferentes áreas do trabalho e o compromisso com o uso diligente delas reflete a constante procura por eficiência e qualidade nos serviços prestados pela Deloitte.


    \subsection{Estrutura Organizacional da RIL na Deloitte PT}\label{sec:estrutura-organizacional-da-ril-na-deloitte-PT}

        A estrutura do projeto da RIL dentro da Deloitte é fundamental para a garantir o seu sucesso e a eficiência, e foi meticulosamente delineada para maximizar os ganhos do trabalho individual no projeto como um todo. 

        \subsubsection{Reuniões KT}

            Uma das ferramentas usadas nesta organização são, por exemplo, as reuniões KT:
            
            \textbf{Reuniões KT}: As Reuniões KT são sessões que se têm em intervalos de tempo irregulares onde os trabalhadores do projeto se juntam todos e é discutida uma nova funcionalidade, de uma parte da aplicação ou uma implementação, onde são encorajadas também perguntas. Estas reuniões são gravadas e disponibilizadas para todos os envolvidos. 

        \subsubsection{Formações Introdutórias}\label{formacoes_introdutorias_empresa_cliente}

            Existem algumas formações introdutórias que cada novo elemento no projeto tem que seguir, as formações a completar são estabelecidas consoante a equipa do recém-chegado. No caso da equipa AMS RUN, os novos membros têm que completar as seguintes formações:
            \begin{itemize}
                \item Formação de OWASP da Udemy através da Deloitte;
                \item \href{https://learn.mongodb.com/learning-paths/introduction-to-mongodb}{Formação introdutória a MongoDB};
                \item \href{https://learn.mongodb.com/courses/mongodb-aggregation}{Formação de aggregations de MongoDB};
                \item Uma lista de demo vídeos acerca do funcionamento da plataforma;
                \item Parte da documentação técnica do projeto.
            \end{itemize}

        \subsubsection{Prioridades e 24/7}

            % todo colocar aqui uma referencia para a ServiceNow ou para onde falares melhor da priorização de incidentes e defects.

            Outra forma de organização é através da categorização de gravidade de incidentes e alertas. Neste contexto, são adotados diferentes níveis de prioridade, representados por P1, P2, P3 e P4 que determinam a urgência de cada incidente onde P1 é o mais crítico. Por vezes há alertas críticos mesmo fora de horas de trabalho, por isso foi necessário a criação de um papel, o 24/7, no qual um indivíduo fica elegível para lidar com incidentes graves a qualquer momento do dia caso apareçam. Esta posição rotativa requer disponibilidade contínua sendo remunerada adequadamente.

        \subsubsection{Equipas}

            A termos de estrutura, dentro da Deloitte PT, a organização da RIL é dividida em grupos e equipas específicas, cada uma com responsabilidades distintas. Destacamos dois principais grupos:

            \subsubsubsection{Grupo RUN}

                O Grupo RUN é responsável pela manutenção da aplicação e correção de incidentes. Este grupo está subdividido em várias equipas, cada uma focada numa área específica:

                \begin{itemize}
                    \item \textbf{Infraestrutura (Azure):} \\
                    Equipas responsáveis pela gestão da infraestrutura na plataforma Azure, incluindo a manutenção dos serviços Microsoft integrados com a plataforma, como as contas usadas para testar e aceder à plataforma, e a sua autenticação, feitas através do Microsoft ActiveDirectory;
                    \item \textbf{MongoDB:} \\
                    Equipas que gerem a base de dados de MongoDB e questões relacionadas ao desempenho e às pesquisas da base de dados;
                    \item \textbf{OutSystems (OS):} \\
                    O ramo de OutSystems está também dividido em várias equipas na Deloitte PT, existem tarefas predefinidas que alternam regularmente entre estas equipas, normalmente de 3 em 3 semanas:
                    
                    \begin{itemize}
                        \item \textbf{Triagem:} \\
                        Equipa focada em incidentes de produção, o \textit{team leader} deverá atribuir uma prioridade aos incidentes e distribuí-los pela equipa;
                        
                        \item \textbf{Equipa \textit{Lights On}:} \\
                        Equipa responsável pela resolução de \textit{market defects}, ou seja, defeitos associados a incidentes de produção. Estão também encarregues, simultaneamente, das tarefas semanais, estas são tarefas como:
                        \begin{itemize}
                            \item Deploys: Passagem de fase de um ambiente para outro de acordo com o \hyperref[fig:deployment-aggregator]{Deployment Agreaggator};
                            %\ref{fig:deployment-agreggator}
                            \item Scans de segurança;
                            \item Scans e questões de desempenho;
                            \item Scans de processos.
                        \end{itemize}
                        
                        \item \textbf{Resolve \textit{defects}:} \\
                        Apenas resolve \textit{defects} não associados a \textit{defects} de produção, ou seja, qualquer defeito detetado em testes, durante o desenvolvimento ou durante a resolução de outros \textit{defects}. São também responsáveis por preparar a nova \textit{release};
                    \end{itemize}

                    Estas equipas estão muito interligadas, e é frequente se misturarem tarefas quando necessário, por exemplo, numa época com alta procura em que haja muitos incidentes de produção é frequente haver alguns membros de algumas equipas a ajudarem na triagem.
                    Para além destas três equipas rotativas, existe também uma quarta equipa temporária:
                    \begin{itemize}
                        \item \textbf{Surge Team}: \\
                            A equipa Surge tem objetivos específicos bem definidos. É uma equipa formada por vários elementos das restantes equipas com diferentes graus de conhecimentos em diferentes matérias permitindo-lhe abranger uma grande amplitude de problemas, o único objetivo desta equipa é resolver \textit{defects}. 
                            
                            A equipa tem uma \text{release} própria e um conjunto de incidentes a resolver previamente acordados com o cliente, tem metas específicas e prazos de entrega definidos. 

                            A criação desta equipa deu-se devido à necessidade constante de pessoal em triagem, o objetivo é a correção de defeitos ou fontes de bugs na aplicação que estão a fazer com que muitos incidentes sejam abertos na triagem, e ao mesmo tempo, baixar o número de incidentes no \textit{backlog} do projeto.

                            %O número geral das suas releases são 2.3.x.
                    \end{itemize}
                \end{itemize}

                
            \subsubsubsection{Grupo Delivery}
            
                A equipa Delivery é responsável pelo desenvolvimento, casos de uso e implementação de novas funcionalidades. Este grupo trabalha de forma estratificada, dividindo-se em PODs, equipas com objetivos específicos, que seguem a metodologia de \textit{sprints} para o lançamento de \textit{releases}. Cada \textit{release}, por exemplo, 2.4, é composta por um conjunto de X \textit{sprints}.
                
                As PODs, cujas tarefas são rotativas, trabalham em \textit{sprints}, focando-se em tarefas como \textit{User Stories}. Antes de iniciar um \textit{sprint}, são realizadas sessões de \textit{Detail Design} (DD \textit{sessions}) para escrever e rever as \textit{User Stories}, garantindo clareza e tirando quaisquer dúvidas que possam existir.
                
                No dia da entrega de uma \textit{User Story} há uma reunião com a equipa de testes (BAs), onde a \textit{User Story} é apresentada conforme o seu ``happy path'' ou o caminho principal, para garantir que esteja a funcionar conforme o esperado.
                
                Algumas equipas específicas no Grupo Delivery incluem a equipa de APIs (foca-se na manutenção dos APIs de MongoDB e Azure) e a Data Team (trabalha com a base de dados MongoDB).
